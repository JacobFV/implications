\section{Architectural Hierarchies of Meaning Generation}

But how exactly does meaning-generation depend upon---and scale with---the intrinsic sophistication of an agent's internal cognitive architecture? Concrete clarity comes when we step beyond abstraction and directly examine artificial agents whose internal structures are explicitly known. By rigorously tracking how structured correlations flow and persist through diverse computational architectures---each with carefully characterized internal memory and representational capacity---we uncover a precise hierarchy of meaning-generation that emerges naturally from information-theoretic principles.

Consider first a simple stateless feedforward policy without internal memory (for instance, a basic Multilayer Perceptron policy network). Such an agent's chosen actions depend strictly on instantaneous observations, with no retention of correlations over time except whatever residuals the environment leaves untouched. After marginalizing out the meaning-structure baked in these residuals, its meaning-generation rate, i.e., the bits of newly introduced correlation per step ($\mathcal{M}_\text{rate}(t) = \left.\frac{d\mathcal{C}}{dt}\right\vert_{\text{agent}}$), is limited by the narrow bottleneck of input--output capacity, ($\mathcal{M}_\text{rate}(t) \leq \min(n_o, n_a)$), never exceeding the immediate observation-action channel capacity ($n_o \rightarrow n_a$). Formally, total meaning generated accumulates only linearly and locally within each step, never preserving any intricate temporal patterns:

\begin{equation*}
\mathcal{M}_{\text{total}}^{\text{stateless}} \leq \sum_t \min(n_o, n_a)
\end{equation*}

Adding recurrence changes this picture. Consider next a finite-state recurrent network (such as an RNN, GRU, or LSTM) with a finite hidden-state dimension ($d$), each unit carrying roughly ($b$) bits, and state-transition dynamics introducing gradual forgetting (governed by a decay factor ($\rho < 1$)). Such architectures allow the agent to carry forward structured correlations from past timesteps, though inveitably exponentially decaying into noise with increasing lag. Their total internal memory reservoir---representing held correlations---thus saturates to a limited horizon explicitly bounded by both state dimensionality and leakage:

\begin{equation*}
\mathcal{C}_{\text{mem}}^{\text{RNN}} \leq \min \left( d b, n_o, \frac{\rho}{1-\rho} \right)
\end{equation*}

Hence, recurrent agents offer meaning-generation budgets exceeding stateless architectures by a finite additive term for memory---but still mandating saturation:

\begin{equation*}
\mathcal{M}_{\text{total}}^{\text{RNN}} \leq \sum_t \min(n_o, n_a) + \mathcal{C}_{\text{mem}}^{\text{RNN}}
\end{equation*}

What about architectures specifically designed around long-range correlation retention? A fixed-length transformer that precisely attends to the last $L$ observations improves markedly by perfectly encoding an extended observational window. Its internal memory reservoir thus grows linearly with window length:

\begin{equation*}
\mathcal{C}_{\text{mem}}^{\text{TF}} \leq L n_o
\end{equation*}

Correspondingly, total meaning accumulation substantially extends beyond simpler recurrence:

\begin{equation*}
\mathcal{M}^{\text{TF}}_{\text{total}} \leq \sum_t \min(n_o,n_a) + L n_o
\end{equation*}

Yet even here, correlation horizons plateau when their fixed-length windows saturate. How might an agent transcend this plateau completely? By incorporating external differentiable memory modules (for example, retrieval-augmented transformers), an agent can preserve correlations elegantly across its entire lifetime trajectory. Such architectures introduce external memorized tables with $N_e$ memory entries of size $n_e$, queried $k$ times per step. Consequently, memory capacity leaps dramatically, scaling indefinitely according to external storage size and engineering choices:

\begin{equation*}
\mathcal{C}_{\text{mem}}^{\text{EXT}} \leq L n_o + k n_e
\end{equation*}

Finally, at the pinnacle sit fully ``agentic'' stacks featuring internal world-model simulation, long-term episodic memory stores that expand continually, and hierarchical goal-directed deliberation (such as Transformer Temporal-Context (TTC) or Transformer Temporal-Reinforcement Learning (TT-RL) agents). Within these designs, structured informational correlations persist---even sharpen---in multiple complementary memory reservoirs: internally coherent simulation parameters ($C_w$), plus episodic memories that expand cumulatively with each timestep ($m$ new entries, each $n_e$ bits, per timestep over lifetime $T$):

\begin{equation*}
\mathcal{C}_{\text{mem}}^{\text{agent}}(T) \leq C_w + m n_e T
\end{equation*}

Aggregating meaning generated over its lifetime, this powerful final class outstrips all architectures reviewed thus far, as its potential expansions scale indefinitely---limited only by engineering and ultimately cosmic constraints:

\begin{equation*}
\mathcal{M}_{\text{total}}^{\text{agent}}(T) \leq \sum_t \min(n_o,n_a) + C_w + m n_e T
\end{equation*}

The hierarchy we uncover here delineates meaning's explicit correlation with architectural complexity: Stateless agents create superficial momentary correlations; RNNs add exponentially decaying memory; Transformers extend memory linearly within finite horizons; External-memory methods establish lasting lifetimes-spanning storage. Finally, high-powered agentic architectures with world-models, episodic memories, and hierarchical reasoning excel profoundly---architecting cumulative trajectories whose informational significance can persist indefinitely against entropy.

These explicit, information-theoretically grounded examples clarify that meaning scales directly with the richness and sophistication of an agent's internal cognitive architecture. Particular designs immensely surpass others in their capacity to weave intricate correlations across space and time, illuminating an evaluative framework not merely theoretical, but rigorously measurable.

TODO: i should not be so conclusive here. After all a transformer is already the kernel of an agent that just needs software 'training wheels' to teach it to actively jog/retain/organize its memory. It needs 
Rather this discussion should've been for establishing the classes of architectures we will be analyziing our measureable qualia operationals on and then making ad-hoc commentary on the meaning of each architectures existance in a given situation. In the retrospective at the end we can make stronger statements about meaning and specific architectural design priors. Although certainly now we can already make comments on teh standard correlation length of each of these architectures from a known initialization. Additionally we can make comments on the correlation length preservation based soley on language benchmark scores

Far from mystical or subjective handwaving, meaning emerges clearly as robustly quantifiable structure---an agent's deliberate imprinting of informational coherence revolting against universal entropy. Equipped with this rigorous clarity, we are finally prepared to approach perhaps the richest and most profound natural expression of structured correlation-preservation underlying human life: love.
