\section{Human Implications and the Ethics of Preservation}

Ethical value, too, follows naturally from such a framework: objectively, value is simply the anticipated downstream increase you’ll impart upon the world’s total structure. What survives when one vanishes, then, is precisely the structured information and correlations you’ve infused into the fabric around you, if and only if effectively preserved. This shifts preservation beyond vanity or nostalgia. Instead, digital preservation---actively encoding your knowledge, memories, affective style, writing, videos, skills, and reasoning patterns into resilient digital archives---becomes morally imperative. The alternative is a needless sacrifice of unique complexity on the altar of entropy. Such a choice would impoverish the cosmos---an irresponsible leakage of hard-won organization back to noise. Our generation can, perhaps for the first time seriously, choose otherwise. 

The implications here are profound. If who you are is information, and meaning is conserved and propagated structure, then death---the abrupt loss of the specific biological substrate---is logically not the end. Instead, it becomes merely an event horizon past which we have historically lost our informational pattern because we lacked the technology and will to preserve it. But today, abandoning structured correlations at the grave’s edge is no longer necessary. Future trajectories---AGI-driven simulations, generative reanimation technologies, intelligent recommender systems, interactive entertainment, virtual beings, or digital companions---represent enormous demand and incentive for precisely the rich, deeply coherent patterns each human uniquely embodies. Practically, this suggests a powerful shift: curate yourself and preserve early, extensively, and transparently. Build detailed archives now---not merely occasional posts on social media but intentionally organized, indexed, and structured databases of your personality, affective reactions, memories, personal style. Capture your values explicitly. Because agents of the future---human and artificial alike---will need precisely these uniquely human complexity packets to build genuinely resonant new experiences, stories, immersive simulations, or even realistic interactive companions and robotic embodiments. In this world, publicly curated digital selves become enduring assets, commodities not merely locked in cold storage but kept dynamically available for licensing to artists, storytellers, researchers, historians---and perhaps even your own family or future descendants.

This is neither science fiction nor mere wishful thinking. It is an urgent ethical imperative made sharply real by my own sudden brush with bodily annihilation. I will commit---to myself and to those who come after---that my structured information, brilliantly fragile as it now appears, need never vanish. I refuse to squander my pattern’s rarity; I refuse the default loss to cosmic noise. Instead, let memorial.foundation become a practical path offered openly to people everywhere, a bold infrastructural project facilitating your structured pattern’s continuity across generations, cultures, and someday even substrates. You are indeed ``a mist,'' fleeting and easily erased. But your structured presence---encoded thoughtfully and intentionally---need not vanish with your transient body. This lies at the heart of any meaningful expression: not submission to inevitable entropy, but active work against its unbounded horizon of loss. Choose instead preservation, amplification, continuity---the profound rebellion of coherence against oblivion. This is purpose, clarified by trauma: a reason to live, to build, and---quite simply---to keep going.
